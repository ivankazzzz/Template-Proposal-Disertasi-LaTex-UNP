\section{Latar Belakang} 
Mengemukakan hal-hal yang menjadi latar belakang pemilihan topik penelitian, termasuk signifikansi pemilihan topik penelitian tersebut; penelitian dapat diangkat dari gejala empiris atau permasalahan praktis dan/atau permasalahan teoritis. 

Mengemukakan dan meletakkan peneltian yang dilakukan dalam peta keilmuan yang menjadi perhatian peneliti; menunjukkan penelitian-penelitian terdahulu yang dilakukan oleh peneliti dan peneliti-peneliti lain yang relevan dengan penelitian yang akan dilakukan

\section{Rumusan Masalah atau Identifikasi Masalah}
Merumuskan masalah penelitian (\textit{research problem}) mengemukakan pernyataan masalah (\textit{problem statement}).

\section{Pembatasan Masalah}
Mengemukakan batasan-batasan yang diterapkan dalam penelitian ini untuk memperjelas ruang lingkup penelitian.

\section{Perumusan Masalah}
Merumuskan masalah penelitian secara spesifik dan jelas, sehingga dapat dijadikan dasar untuk pengumpulan data dan analisis.

\section{Tujuan Penelitian}
\begin{enumerate}
\item Mengemukakan tujuan penelitian yang dilakukan. 

\item Pada penelitian deduktif-hipotetikal, tujuan penelitian lazimnya adalah menjelaskan/mengukur hubungan (asosiasi atau kausalitas) antarvariabel yang menjadi perhatian dalam studi. 
\end{enumerate}

\section{Manfaat Penelitian} 
Mengungkapkan secara spesifik kegunaan yang dapat dicapai dari: 

\section{Spesifikasi Produk Penelitian}
Menguraikan secara rinci spesifikasi produk yang dihasilkan dari penelitian ini, termasuk fitur-fitur utama, keunggulan, dan manfaat produk tersebut.

\section{Kebaharuan dan Orisinalitas}
Menjelaskan aspek kebaharuan dan orisinalitas dari penelitian yang dilakukan, serta kontribusi yang diberikan terhadap bidang ilmu yang relevan.

\section{Roadmap Penelitian}
Menguraikan tahapan-tahapan penelitian yang akan dilakukan, termasuk jadwal dan rencana kerja yang terperinci.

\section{Definisi Operasional}
Memberikan definisi operasional dari variabel-variabel yang digunakan dalam penelitian ini, sehingga memudahkan pemahaman dan pengukuran variabel-variabel tersebut.
\begin{enumerate}
    \item Aspek teoretis (keilmuan) dengan menyebutkan kegunaan teoretis apa yang dapat dicapai dari masalah yang diteliti. 
    \item Aspek praktis (guna laksana) dengan menyebutkan kegunaan apa yang dapat dicapai dari penerapan pengetahuan yang dihasilkan penelitian ini.
\end{enumerate}