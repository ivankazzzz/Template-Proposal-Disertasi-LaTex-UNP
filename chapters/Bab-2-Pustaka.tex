Secara umum bab ini, menguraikan paradigma/pendekatan/metode yang dipergunakan dalam penelitian. Uraian mencakup, tapi tidak terbatas pada, hal-hal sebagai berikut:
\begin{itemize} 
\item Uraian tentang rancangan penelitian yang dipilih.
\item Prosedur pengambilan/pemilihan sampel dan penentuan unit analisis.
\item Sumber dan teknik pengumpulan data serta instrumen penelitian. 
\item Pengolahan dan analisis data termasuk (uji) validitas data yang sesuai dengan rancangan penelitian yang diusulkan. 
\item Lokasi dan waktu penelitian. 
\end{itemize}

Pada beberapa disiplin di bidang ilmu-ilmu eksakta, bab ini diberi judul "BAHAN/OBJEK DAN METODE PENELITIAN". Sesuai dengan judul tersebut, uraian pada bab ini dimulai dengan uraian tentang bahan, subjek, dan objek penelitian di dalam bagian yang diberi sub-judul "Bahan/Objek Penelitian". Kemudian dilanjutkan dengan uraian yang diberi sub-judul "Metode Penelitian"; uraian membuat butir-butir seperti pada paradigma/pendekatan/metode di atas.

\section{Jenis Penelitian}
\textbf{Contoh Penulisan}. Penelitian ini menggunakan metode penelitian deskriptif dengan pendekatan kuantitatif. Metode ini dipilih karena sesuai dengan tujuan penelitian yang ingin menggambarkan fenomena yang terjadi secara sistematis dan akurat.

\section{Prosedur Penelitian}
\textbf{Contoh Penulisan}. Prosedur penelitian ini meliputi beberapa tahap, yaitu: (1) persiapan, (2) pelaksanaan, dan (3) analisis data. Pada tahap persiapan, dilakukan pengumpulan literatur dan penyusunan instrumen penelitian. Tahap pelaksanaan melibatkan pengumpulan data di lapangan. Tahap analisis data dilakukan dengan menggunakan teknik statistik yang sesuai.

\section{Subjek Penelitian}
\textbf{Contoh Penulisan}. Subjek penelitian ini adalah siswa kelas X di SMA Negeri 1 Kota X. Pemilihan subjek dilakukan secara acak dengan menggunakan teknik sampling acak sederhana.

\section{Instrumen dan Teknik Pengumpulan Data}
\textbf{Contoh Penulisan}. Instrumen yang digunakan dalam penelitian ini adalah angket dan lembar observasi. Angket digunakan untuk mengumpulkan data tentang persepsi siswa, sedangkan lembar observasi digunakan untuk mengamati perilaku siswa selama proses pembelajaran. Teknik pengumpulan data dilakukan dengan cara menyebarkan angket kepada siswa dan melakukan observasi langsung di kelas.

\section{Teknik Analisis Data}
\textbf{Contoh Penulisan}. Data yang telah dikumpulkan dianalisis dengan menggunakan teknik analisis statistik deskriptif dan inferensial. Analisis deskriptif digunakan untuk menggambarkan data secara umum, sedangkan analisis inferensial digunakan untuk menguji hipotesis penelitian.


