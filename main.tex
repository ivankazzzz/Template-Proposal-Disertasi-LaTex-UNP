\documentclass[12pt, a4paper, onecolumn, oneside, final]{report}
\usepackage[utf8]{inputenc}
%\usepackage{fontspec}
%\setmainfont{Times New Roman}
%%%%
\usepackage{tocbibind} %%% untuk memasukkan bibliography pada daftar isi
%
%%%%%%  adding image %%%
\usepackage{graphicx}
\graphicspath{{images/}}
\newcommand{\pic}{Gambar}
%%%%
\usepackage{setspace} % pengaturan line spacing 
%%%%%% 
\usepackage[paper=a4paper,headheight=0pt,left=4cm,top=4cm,right=3cm,bottom=3cm]{geometry}
\usepackage[font=footnotesize,format=plain,labelfont=bf,up,textfont=up]{caption} 
\captionsetup{labelsep=space}
\usepackage{amsmath}
\usepackage{amsfonts}
\usepackage{bm}
%%%%%
\usepackage{pslatex} % untuk menggunakan font times new roman
%%%%%%
\usepackage{fancyhdr}
%%%%%%% Pengaturan header dan footer dalam dokumen.
\fancyhf{} 
\fancyhead[R]{\thepage}
\renewcommand{\headrulewidth}{0.0pt}
\pagestyle{fancy}
\setlength{\headheight}{14pt}
\addtolength{\topmargin}{-2pt}
%%%%%%%% pengaturan tabel dan gambar
\usepackage{float}
  \floatplacement{figure}{H}
  \floatplacement{table}{H}
%%%%%%%%  
\usepackage{pdfpages} %% untuk memasukkan pdf pada dokumen 
%%%%%%% Konfigurasi Daftar Pustaka
\usepackage{natbib}
\setcitestyle{citesep={;}, aysep={,}} % tambah semikolon dan koma di sitasi
\renewcommand\harvardyearleft{ \unskip } % remove parantheses di dapus
\renewcommand\harvardyearright[1]{.} % remove parantheses di dapus
%%%%%
\usepackage[bahasai]{babel} % Paket bahasa indonesia dan inggris
%%%%%%
%%%%%% konfigurasi setiap subbab
\usepackage{titlesec}
\titleformat{\section}
  {\normalfont\fontsize{12}{12}\bfseries}
  {\thesection}
  {1em}
  {} 
\titleformat{\subsection}
  {\normalfont\fontsize{12}{12}\bfseries}
  {\thesubsection}
  {1em}
  {} 
%%%%%%% Konfigurasi untuk tajuk bab %%%%%%%%%%%%%%%%%%%%%%%%%%%%%%%%%%%%%%%%%%
\makeatletter
\def\@makechapterhead#1{%
  %%%%\vspace*{50\p@}% %%% removed! % <----------------- Space from top of page to Chapter #
  {\parindent \z@ \centering \normalfont
    \ifnum \c@secnumdepth >\m@ne
        \large\bfseries \MakeUppercase{\@chapapp}\space \thechapter % <--- uppercase
        \par\nobreak
        \vskip 5\p@ %<-------------- Space between Chapter # and title
    \fi
    \interlinepenalty\@M
    \large \bfseries #1\par\nobreak
    \vskip 30\p@
  }}
%
\def\@makeschapterhead#1{% % format tulisan "daftar isi" 
  %%%%%\vspace*{50\p@}% %%% removed! % <----------------- Space from top of page to Chapter #
  {\parindent \z@ \centering
    \normalfont
    \interlinepenalty\@M
    \large\bfseries \MakeUppercase  #1\par\nobreak
    \vskip 30\p@
 }}
\makeatother
%%%%%%%%%%%%%%%%%%%%%%%%%%%%%%%%%%%%%%%%%%%%%%%%%%%%%%%%%%%%%%%%%%%%%%%%%%%%%
\usepackage{indentfirst} % Indentasi paragraf pertama
\setlength{\parindent}{1.2cm}
%%%%%%%%%%%%%%%%%%%%%%%%Paket untuk menggambar senyawa kimia%%%%
\usepackage{chemfig}
\usepackage[version=4]{mhchem}
\usepackage{chemnum}
\makeatletter
\def\Hv@scale{0.65}
\makeatother
\DeclareMathAlphabet{\foo}{OT1}{phv}{m}{n}
\renewcommand*\printatom[1]{\ensuremath{\foo{#1}}}
\makeatother
\setchemfig{double bond sep = 0.20700 em,  % 'Bond Spacing' 
            fixed length = false,    % 'Fixed Length'
            bond offset = 0.18265 em, % 'Margin Width'
            bond style={line width=0.50pt},
            atom sep = 1.2 em}
%%%%%%%%%%%%%%%%%%%%%%%%%%%%%%%%%%%%%%%%%%%%%%%%%%%%%%%
\usepackage[hidelinks]{hyperref}
%%%%%Table of Contents typography%%%%%%%%%%%%%%%%%%%%%%
\usepackage[titles]{tocloft}
\newlength\mylength
\renewcommand\cftchappresnum{\chaptername~}
\settowidth\mylength{\cftchappresnum\cftchapaftersnum\quad}
\addtolength\cftchapnumwidth{2.3em}
%%%%%%%%%%%%%%%%%%%%%%%%%%%%Paket untuk pengaturan tabel%%%%%%%%%%
\usepackage{booktabs}
\usepackage{multirow}
\usepackage{colortbl}
%%%%%%%%%%%%%%%%%%%%%%%%%%%%Paket untuk penulisan Satuan Internasional%%%
\usepackage{siunitx}
\DeclareSIUnit{\Molar}{M}
\DeclareSIUnit{\calorie}{cal}
\DeclareSIUnit{\Calorie}{\kilo\calorie}
%%%%%%%%%%%%%%%%%%%%%%%%%%%%%%%%%%%%%%%%%%%%%%%%%%%%%
%%%%%%%%%%-----KONTEN PENELITIAN-----%%%%%%%%%%%%%%%%%%%%%%%%%%%%%%%%%%%%%%%%
\begin{document}

\singlespacing
\begin{titlepage}
  \begin{center}
      
      \Large
        \textbf{JUDUL TESIS atau DISERTASI}
                
        \vspace{2.0cm} % Adjust the space between the title and the proposal text
                 
        \large
        PROPOSAL DISERTASI \\ \small
        \textit{Untuk Memenuhi Sebagian Persyaratan Mencapai Derajat Doktor \\
            Program Studi Pendidikan Ilmu Pengetahuan Alam}           
        \vfill
            
        \includegraphics[width=6.0cm]{front/logo-unp.png}
            
        \vfill
        \large
        \textbf{Oleh} \\
        \textbf{Irfan Ananda Ismail} \\
        \textbf{24341017}
            
        \vfill
            
        \large
        \textbf{PROGRAM DOKTOR PENDIDIKAN IPA \\
        FAKULTAS MATEMATIKA DAN ILMU PENGETAHUAN ALAM\\
        UNIVERSITAS NEGERI PADANG \\
        2024}
      
  \end{center}
\end{titlepage}

\pagenumbering{roman} %Gunakan penomoran romawi

\setcounter{page}{1}
\singlespacing
\setlength{\cftbeforechapskip}{0pt} % Remove space between chapter entries
\renewcommand{\cftdot}{.} % Define dot style
\renewcommand{\cftdotsep}{2} % Define dot separation

\chapter*{HALAMAN JUDUL}\addcontentsline{toc}{chapter}{HALAMAN JUDUL}\input{chapters/HalamanJudul}
\chapter*{PERSETUJUAN AKHIR DISERTASI}\addcontentsline{toc}{chapter}{PERSETUJUAN AKHIR DISERTASI}\input{chapters/PersetujuanAkhir}
\chapter*{PERSETUJUAN KOMISI UJIAN DISERTASI}\addcontentsline{toc}{chapter}{PERSETUJUAN KOMISI UJIAN DISERTASI}\input{chapters/PersetujuanKomisi}
\chapter*{PERNYATAAN KEASLIAN KARYA TULIS DISERTASI}\addcontentsline{toc}{chapter}{PERNYATAAN KEASLIAN KARYA TULIS DISERTASI}\input{chapters/PernyataanKeaslian}
\chapter*{KATA PENGANTAR}\addcontentsline{toc}{chapter}{KATA PENGANTAR}\input{chapters/KataPengantar}

\singlespacing
\addcontentsline{toc}{chapter}{DAFTAR ISI}
\tableofcontents
\listoffigures
\listoftables
\clearpage

\chapter*{DAFTAR LAMPIRAN}\addcontentsline{toc}{chapter}{DAFTAR LAMPIRAN}\input{chapters/DaftarLampiran}
\chapter*{ABSTRAK}\addcontentsline{toc}{chapter}{ABSTRAK}\noindent Mencerminkan seluruh isi tesis/disertasi dengan mengungkapkan
intisari permasalahan penelitian, pendekatan yang digunakan atau
kerangka pemikiran, metode penelitian, temuan penelitian, dan
kesimpulan. Uraian ditulis dalam Bahasa
Indonesia, masing-masing tidak lebih dari 500 kata. 

\vspace{5mm}

\noindent Kata kunci: 
\chapter*{ABSTRACT}\addcontentsline{toc}{chapter}{ABSTRACT}\noindent \textit{Mencerminkan seluruh isi tesis/disertasi dengan mengungkapkan
intisari permasalahan penelitian, pendekatan yang digunakan atau
kerangka pemikiran, metode penelitian, temuan penelitian, dan
kesimpulan. Uraian ditulis dalam Bahasa Inggris, masing-masing tidak lebih dari 500 kata dan ditulis italic}.

\vspace{5mm}

\noindent \textit{Keywords:}   

\pagenumbering{arabic} %Gunakan penomoran arab
\doublespacing
\renewcommand{\thechapter}{\Roman{chapter}}
\renewcommand{\thesection}{\arabic{chapter}.\arabic{section}}
\renewcommand{\thefigure}{\arabic{chapter}.\arabic{figure}}
\renewcommand{\thetable}{\arabic{chapter}.\arabic{table}}
\chapter{PENDAHULUAN} 
\section{Latar Belakang} 
Mengemukakan hal-hal yang menjadi latar belakang pemilihan topik penelitian, termasuk signifikansi pemilihan topik penelitian tersebut; penelitian dapat diangkat dari gejala empiris atau permasalahan praktis dan/atau permasalahan teoritis. 

Mengemukakan dan meletakkan peneltian yang dilakukan dalam peta keilmuan yang menjadi perhatian peneliti; menunjukkan penelitian-penelitian terdahulu yang dilakukan oleh peneliti dan peneliti-peneliti lain yang relevan dengan penelitian yang akan dilakukan

\section{Rumusan Masalah atau Identifikasi Masalah}
Merumuskan masalah penelitian (\textit{research problem}) mengemukakan pernyataan masalah (\textit{problem statement}).

\section{Pembatasan Masalah}
Mengemukakan batasan-batasan yang diterapkan dalam penelitian ini untuk memperjelas ruang lingkup penelitian.

\section{Perumusan Masalah}
Merumuskan masalah penelitian secara spesifik dan jelas, sehingga dapat dijadikan dasar untuk pengumpulan data dan analisis.

\section{Tujuan Penelitian}
\begin{enumerate}
\item Mengemukakan tujuan penelitian yang dilakukan. 

\item Pada penelitian deduktif-hipotetikal, tujuan penelitian lazimnya adalah menjelaskan/mengukur hubungan (asosiasi atau kausalitas) antarvariabel yang menjadi perhatian dalam studi. 
\end{enumerate}

\section{Manfaat Penelitian} 
Mengungkapkan secara spesifik kegunaan yang dapat dicapai dari: 

\section{Spesifikasi Produk Penelitian}
Menguraikan secara rinci spesifikasi produk yang dihasilkan dari penelitian ini, termasuk fitur-fitur utama, keunggulan, dan manfaat produk tersebut.

\section{Kebaharuan dan Orisinalitas}
Menjelaskan aspek kebaharuan dan orisinalitas dari penelitian yang dilakukan, serta kontribusi yang diberikan terhadap bidang ilmu yang relevan.

\section{Roadmap Penelitian}
Menguraikan tahapan-tahapan penelitian yang akan dilakukan, termasuk jadwal dan rencana kerja yang terperinci.

\section{Definisi Operasional}
Memberikan definisi operasional dari variabel-variabel yang digunakan dalam penelitian ini, sehingga memudahkan pemahaman dan pengukuran variabel-variabel tersebut.
\begin{enumerate}
    \item Aspek teoretis (keilmuan) dengan menyebutkan kegunaan teoretis apa yang dapat dicapai dari masalah yang diteliti. 
    \item Aspek praktis (guna laksana) dengan menyebutkan kegunaan apa yang dapat dicapai dari penerapan pengetahuan yang dihasilkan penelitian ini.
\end{enumerate} 

\chapter{KAJIAN PUSTAKA}
Secara umum bab ini, menguraikan paradigma/pendekatan/metode yang dipergunakan dalam penelitian. Uraian mencakup, tapi tidak terbatas pada, hal-hal sebagai berikut:
\begin{itemize} 
\item Uraian tentang rancangan penelitian yang dipilih.
\item Prosedur pengambilan/pemilihan sampel dan penentuan unit analisis.
\item Sumber dan teknik pengumpulan data serta instrumen penelitian. 
\item Pengolahan dan analisis data termasuk (uji) validitas data yang sesuai dengan rancangan penelitian yang diusulkan. 
\item Lokasi dan waktu penelitian. 
\end{itemize}

Pada beberapa disiplin di bidang ilmu-ilmu eksakta, bab ini diberi judul "BAHAN/OBJEK DAN METODE PENELITIAN". Sesuai dengan judul tersebut, uraian pada bab ini dimulai dengan uraian tentang bahan, subjek, dan objek penelitian di dalam bagian yang diberi sub-judul "Bahan/Objek Penelitian". Kemudian dilanjutkan dengan uraian yang diberi sub-judul "Metode Penelitian"; uraian membuat butir-butir seperti pada paradigma/pendekatan/metode di atas.

\section{Jenis Penelitian}
\textbf{Contoh Penulisan}. Penelitian ini menggunakan metode penelitian deskriptif dengan pendekatan kuantitatif. Metode ini dipilih karena sesuai dengan tujuan penelitian yang ingin menggambarkan fenomena yang terjadi secara sistematis dan akurat.

\section{Prosedur Penelitian}
\textbf{Contoh Penulisan}. Prosedur penelitian ini meliputi beberapa tahap, yaitu: (1) persiapan, (2) pelaksanaan, dan (3) analisis data. Pada tahap persiapan, dilakukan pengumpulan literatur dan penyusunan instrumen penelitian. Tahap pelaksanaan melibatkan pengumpulan data di lapangan. Tahap analisis data dilakukan dengan menggunakan teknik statistik yang sesuai.

\section{Subjek Penelitian}
\textbf{Contoh Penulisan}. Subjek penelitian ini adalah siswa kelas X di SMA Negeri 1 Kota X. Pemilihan subjek dilakukan secara acak dengan menggunakan teknik sampling acak sederhana.

\section{Instrumen dan Teknik Pengumpulan Data}
\textbf{Contoh Penulisan}. Instrumen yang digunakan dalam penelitian ini adalah angket dan lembar observasi. Angket digunakan untuk mengumpulkan data tentang persepsi siswa, sedangkan lembar observasi digunakan untuk mengamati perilaku siswa selama proses pembelajaran. Teknik pengumpulan data dilakukan dengan cara menyebarkan angket kepada siswa dan melakukan observasi langsung di kelas.

\section{Teknik Analisis Data}
\textbf{Contoh Penulisan}. Data yang telah dikumpulkan dianalisis dengan menggunakan teknik analisis statistik deskriptif dan inferensial. Analisis deskriptif digunakan untuk menggambarkan data secara umum, sedangkan analisis inferensial digunakan untuk menguji hipotesis penelitian.




\chapter{METODE PENELITIAN}
\input{chapters/Bab-3-Metodologi}

%%%%%%%%%-----DAFTAR PUSTAKA-------%%%%%%%%%%%%%%%%%%%%%%%%%%%%%%%%%%%%%%%%%%%%%
\singlespacing
\bibliographystyle{mydcu-rev} %% modifikasi gaya sitasi dari dcu
\renewcommand{\bibname}{Daftar Pustaka} % ubah "bibliography" menjadi "Daftar Pustaka" 
\bibliography{contoh_pustaka}

\end{document}